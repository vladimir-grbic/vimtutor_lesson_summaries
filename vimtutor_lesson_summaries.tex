\documentclass[handout, 10pt]{beamer}

% Handout settings
\usepackage{pgfpages}
\pgfpagesuselayout{4 on 1}[letterpaper, border shrink=5mm, landscape]

% Beamer settings
\usetheme{default}
\usecolortheme{dove}
\setbeamertemplate{itemize items}[circle]
\setbeamertemplate{enumerate items}[default]

% Format settings
% Set left margin of all list to 0pt
\setlength{\leftmargini}{0pt}
% Remove space before table item (leftmargin=* equivalent)
\renewcommand{\tabcolsep}{0pt}

% Fonts
\usepackage[utf8]{inputenc}
\usepackage[T1]{fontenc}
\usepackage[english]{babel}

% Graphics and objects
\usepackage{xcolor}
\usepackage{graphicx}
\usepackage{float}

% Mathematics
\usepackage{amsmath, amsfonts, amsthm}

% n^{th}
\usepackage[super]{nth} % Example: \nth{1}, \nth{2}, \nth{3}, \nth{4}

% Hyperlink Setup
\usepackage{hyperref}
\urlstyle{same} % Normal looking URL
\hypersetup{
	colorlinks=true,
	linkcolor=black,
	filecolor=magenta,
	urlcolor=black % cyan,
}

% Code Environment
\definecolor{codegray}{gray}{0.9}
\newcommand{\code}[2][black]{\textcolor{#1}{\colorbox{codegray}{\texttt{#2}}}}

% Opening
\title{VIMTUTOR LESSON SUMMARIES}
\author{Vladimir Grbić}
\institute{\href{https://vladimirgrbic.com}{vladimirgrbic.com}}
\date{} % January 15, 2022 <- May 24, 2021 <- March 20, 2021

\begin{document}

\begin{frame}
	\titlepage
\end{frame}

\begin{frame}{Lesson 1 Summary}
	\begin{enumerate}
		\item The cursor is moved using either the arrow keys or the
			\code{hjkl} keys. \\
			\code{h} (left) $ \quad $ \code{j} (down) $ \quad $ \code{k} (up) $
			\quad $ \code{l} (right)

		\item To start Vim from the shell prompt type: \code{vim FILENAME
			<ENTER>}

		\item To exit Vim type: \code{<ESC> :q! <ENTER>} to trash all
			changes. \\
			OR type: \code{<ESC> :wq <ENTER>} to save the changes.

		\item To delete the character at the cursor type: \code{x}

		\item To insert or append text type: \\
			\code{i type inserted text <ESC>} insert before the cursor \\
			\code{A type appended text <ESC>} append after the line
	\end{enumerate}

	\begin{block}{Note}
		Pressing \code{<ESC>} will place you in Normal mode or will cancel an
		unwanted and partially completed command.
	\end{block}
\end{frame}

\begin{frame}{Lesson 2 Summary}
	\begin{enumerate}
		\item To delete from the cursor up to the next word type: \code{dw}

		\item To delete from the cursor to the end of a line type: \code{d\$}

		\item To delete a whole line type: \code{dd}

		\item To repeat a motion prepend it with a number: \code{2w}

		\item The format for a change command is: \\
		\begin{center}
			\code{operator  [number]  motion}
		\end{center}
		where:
		\begin{itemize}
			\item operator - is what to do, such as \code{d} for delete

			\item \text{[number]} - is an optional count to repeat the motion

			\item motion   - moves over the text to operate on, such as
				\code{w} (word), \code{\$} (to the end of line), etc.
		\end{itemize}

		\item To move to the start of the line use a zero: \code{0}

		\item To undo previous actions, type: \code{u} (lowercase u) \\
			To undo all the changes on a line, type: \code{U} (capital U) \\
			To undo the undo's, type: \code{CTRL-R}
	\end{enumerate}
\end{frame}

\begin{frame}{Lesson 3 Summary}
	\begin{enumerate}
		\item To put back text that has just been deleted, type \code{p}. This
			puts the	deleted text AFTER the cursor (if a line was deleted it
			will go on the line below the cursor).

		\item To replace the character under the cursor, type \code{r} and then
			the character you want to have there.

		\item The change operator allows you to change from the cursor to where
			the	motion takes you. \\
			eg. Type \code{ce} to change from the cursor to the end of the
			word, \code{c\$} to change to the end of a line, or \code{cc} to
			change the whole line.

		\item  The format for change is: \\
			\code{c  [number]  motion}
	\end{enumerate}
\end{frame}

\begin{frame}{Lesson 4 Summary}
	\begin{enumerate}
		\item
			\begin{tabular}[t]{rl}
				\code{CTRL-G} & ~displays your location in the file and the
				file status. \\
				\code{G} & ~moves to the end of the file. \\
				\code{number G} & ~moves to that line number.\\
				\code{gg} & ~moves to the first line.
			\end{tabular}

		\item Typing \code{/} followed by a phrase searches FORWARD for the
			phrase.	\\
			Typing \code{?} followed by a phrase searches BACKWARD for the
			phrase. \\
			After a search type \code{n} to find the next occurrence in the
			same direction or \code{N} to search in the opposite direction. \\
			\code{CTRL-O} takes you back to older positions, \code{CTRL-I} to
			newer positions.

		\item Typing  \code{\%}  while the cursor is on a (,),[,],\{, or \}
			goes to its match.

		\item To substitute new for the first old in a line type
			\code{:s/old/new} \\
			To substitute new for all `old's on a line type \code{:s/old/new/g}
			\\
			To substitute phrases between two line \#'s type
			\code{:\#,\#s/old/new/g} \\
			To substitute all occurrences in the file type
			\code{:\%s/old/new/g} \\
			To ask for confirmation each time add `c' \code{:\%s/old/new/gc}
	\end{enumerate}
\end{frame}

\begin{frame}{Lesson 5 Summary}
	\begin{enumerate}
		\item \code{:!command} executes an external command. \\[6pt]
		Some useful examples are:
		\begin{table}
			\begin{tabular}{lll}
				(Windows) & ~(Unix) & \\
				\code{:!dir} & ~\code{:!ls} & ~-- shows a directory listing. \\
				\code{:!del FILENAME} & ~\code{:!rm FILENAME} & ~-- removes
				file FILENAME.
			\end{tabular}
		\end{table}

		\item \code{:w FILENAME} writes the current Vim file to disk with name
			FILENAME.

		\item \code{v  motion  :w FILENAME} saves the Visually selected lines
			in file FILENAME.

		\item \code{:r FILENAME}  retrieves disk file FILENAME and puts it
			below the cursor position.

		\item \code{:r !dir}  reads the output of the \code{dir} command and
			puts it below the cursor position.
	\end{enumerate}
\end{frame}

\begin{frame}{Lesson 6 Summary}
	\begin{enumerate}
		\item Type \code{o} to open a line BELOW the cursor and start Insert
			mode. \\
			Type \code{O} to open a line ABOVE the cursor.

		\item Type  \code{a}  to insert text AFTER the cursor. \\
			Type \code{A} to insert text after the end of the line.

		\item The \code{e} command moves to the end of a word.

		\item The \code{y} operator yanks (copies) text, \code{p} puts (pastes)
			it.

		\item Typing a capital \code{R} enters Replace mode until \code{<ESC>}
			is pressed.

		\item Typing ``\code{:set xxx}'' sets the option ``xxx''. Some options
			are:
			\begin{table}[H]
				\centering
				\begin{tabular}{llcl}
					`ic' & ~`ignorecase' & ~~~ &~ignore upper/lower case when
					searching \\
					`is' & ~`incsearch' & ~~~ &~show partial matches for a
					search phrase \\
					`hls' & ~`hlsearch' & ~~~ &~highlight all matching phrases
				\end{tabular}
			\end{table}
			You can either use the long or the short option name.

		\item Prepend ``no'' to switch an option off: \code{:set noic}
	\end{enumerate}

%	\begin{block}{Note}
%		If you want to ignore case for just one search command, use
%		\code{\textbackslash c} in the phrase: \code{/ignore\textbackslash c
%		<ENTER>}
%	\end{block}
\end{frame}

\begin{frame}{Lesson 7 Summary}
	\begin{enumerate}
		\item Type \code{:help} or press \code{<F1>} or \code{<Help>}  to open
			a help window.

		\item Type \code{:help cmd} to find help on \code{cmd}.

		\item Type \code{CTRL-W CTRL-W} to jump to another window.

		\item Type \code{:q} to close the help window.

		\item Create a vimrc startup script to keep your preferred settings.

		\item When typing a \code{:} command, press \code{CTRL-D} to see
			possible completions. \\
			Press \code{<TAB>} to use one completion.
	\end{enumerate}
\end{frame}

\end{document}
